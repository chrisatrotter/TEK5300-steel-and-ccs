\documentclass{beamer}
\usepackage[utf8]{inputenc}
\usepackage{amsmath, amssymb}
\usepackage{graphicx}
\usepackage{booktabs}
\usepackage{xcolor}
\usepackage{url}
\usepackage[backend=biber,style=authoryear,sorting=none]{biblatex}
\addbibresource{references.bib}

\usetheme{Madrid}
\usecolortheme{beaver}
\setbeamertemplate{navigation symbols}{}
\setbeamertemplate{footline}[frame number]

\title{Steel and CCS:\\A Practical Path to Cut Emissions Today}
\author{Elias Julian Grindstrand \quad
  Farah Said Omar \\
  Christopher A. Trotter \quad
  Sander Finnset Ørnes}
\institute{TEK5300 -- Sustainable Materials \\ University of Oslo}
\date{\today}

\begin{document}

%====================================
\frame{\titlepage}
%====================================

%====================================
\begin{frame}{Steel: Strong, Essential — and a Climate Problem}
\centering
\Large \textbf{Steel is everywhere}\\
\vspace{0.5cm}
\small
\begin{itemize}
    \item Wind turbines, EVs, buildings, bridges
    \item \textbf{100\% recyclable}, \textbf{strong}, \textbf{flexible}
    \item But: \textbf{11\% of global CO$_2$} (SteelWatch, 2025)
\end{itemize}
\end{frame}
%====================================

%====================================
\begin{frame}{How Much Steel?}
\small
\begin{itemize}
    \item \textbf{1.885 billion tonnes} in 2024
    \item \textbf{250,000 Eiffel Towers}
\end{itemize}
\end{frame}
%====================================

%====================================
\begin{frame}{Steel = 11\% of Global CO$_2$}
\centering
\Large \textbf{2.33 t CO$_2$ per tonne of steel}\\
\vspace{0.5cm}
\small
\begin{itemize}
    \item Total: \textbf{3.7 billion tonnes CO$_2$/year}
    \item \textbf{7\% direct} + \textbf{4\% indirect (power)}
    \item \textbf{90\% from coal-based blast furnaces}
\end{itemize}
\pause
\vspace{0.5cm}
\alert{We \textbf{must} change how iron is made}
\end{frame}
%====================================

%====================================
\begin{frame}{How Is Steel Made? (Simplified)}
\centering
\includegraphics[width=0.8\linewidth]{steel_processing_simple.png}\\
\tiny \cite{SteelWatch2025_climate}
\end{frame}

\begin{frame}{How Is Steel Made? (BFF-BOF vs. EAF)}
\small
\begin{columns}[T] % Align top of text and image
    \begin{column}{0.6\textwidth}
        \begin{itemize}
            \item \textbf{70\%}: \alert{BF-BOF} (coal + iron ore $\rightarrow$ steel)
            \begin{itemize}
                \item Coke heats + reduces iron ore
                \item $\rightarrow$ CO$_2$ + molten iron
            \end{itemize}
            \item \textbf{30\%}: \alert{EAF} (electricity + scrap $\rightarrow$ steel)
            \begin{itemize}
                \item Recycles old steel
                \item Lower CO$_2$ if green power
            \end{itemize}
        \end{itemize}
    \end{column}

    \begin{column}{0.4\textwidth}
        \centering
        \includegraphics[width=\linewidth]{steel_bff_bof_eaf_processing_simplified.png}\\
        \tiny \cite{SteelWatch2025_climate}
    \end{column}
\end{columns}
\end{frame}

%====================================

%====================================
\begin{frame}{Where Do Emissions Come From?}
\centering
\small

\begin{tabular}{lcc}
\toprule
\textbf{Stage} & \textbf{CO$_2$ (t/t steel)} \\
\midrule
Coke making & 0.71 \\
Blast furnace & 1.41 \\
Steel refining & 0.21 \\
\bottomrule
\textbf{Total} & \textbf{2.33} \\
\end{tabular}
\vspace{0.2cm}

\tiny \cite{SteelWatch2025_climate}

\vspace{0.4cm}

\alert{\textbf{90\% from blast furnace (BFG)}}

\end{frame}

%====================================

%====================================
\begin{frame}{Solution 1: CCS on Blast Furnace Gas (BFG)}
\centering
\Large \textbf{Capture CO$_2$ from factory exhaust}\\
\vspace{0.5cm}
\small
\begin{block}{Blast Furnace Gas (BFG)}
\begin{itemize}
    \item \textbf{22\% CO$_2$}, 49\% N$_2$, 22\% CO
    \item \textbf{High CO$_2$ = easy to capture}
\end{itemize}
\end{block}
\end{frame}
%====================================

%====================================
\begin{frame}{CCS Challenge: CO$_2$ Must Be Pure for Transport}
\centering
\small
\begin{table}
\begin{tabular}{lcc}
\toprule
\textbf{Impurity} & \textbf{In BFG Capture} & \textbf{Max Allowed (EU)} \\
\midrule
CO & 200–5,000 ppm & 100–2,000 ppm \\
H$_2$S & 40–70 ppm & 5–80 ppm \\
COS & 130–200 ppm & 0.1–10 ppm \\
H$_2$O & Saturated & 30–50 ppm \\
\bottomrule
\end{tabular}
\end{table}
\tiny TCCS-11 (2021), Northern Lights, Porthos
\pause
\vspace{0.3cm}
\alert{\textbf{Extra purification needed}}
\end{frame}
%====================================

%====================================
\begin{frame}{How to Purify CO$_2$ from BFG?}
\centering
\small
\begin{enumerate}
    \item \textbf{Capture}: Amine or PSA $\rightarrow$ 83–99.7\% CO$_2$
    \item \textbf{Remove CO, H$_2$S, COS}:
       \begin{itemize}
           \item \alert{ZnO adsorption} (H$_2$S, COS)
           \item \alert{Catalytic oxidation} (CO $\rightarrow$ CO$_2$)
           \item \alert{Cryogenic distillation} (high purity)
       \end{itemize}
    \item \textbf{Dry}: Molecular sieve $\rightarrow$ <50 ppm H$_2$O
\end{enumerate}
\pause
\vspace{0.3cm}
\alert{Cost: \textbf{+5–15 €/t CO$_2$} for purification}
\end{frame}
%====================================

%====================================
\begin{frame}{Case Study: SSAB Luleå — Partial CCS}
\centering
\includegraphics[width=0.8\linewidth]{lulea_simple.pdf}\\
\tiny Biermann et al., 2019 + WEF 2024
\vspace{0.3cm}
\small
\begin{itemize}
    \item 3.4 Mt CO$_2$/yr $\rightarrow$ \alert{550 GJ/h waste heat}
    \item \textbf{Partial CCS on BFG}: 90\% capture
    \item \textbf{Cuts 36\% site emissions}
    \item \textbf{Cost}: \alert{28 €/t CO$_2$} (uses waste heat!)
\end{itemize}
\end{frame}
%====================================

%====================================
\begin{frame}{Partial CCS = Cheapest & Fastest}
\centering
\small
\begin{table}
\begin{tabular}{lcc}
\toprule
\textbf{Option} & \textbf{CO$_2$ Cut} & \textbf{Cost (€/t)} \\
\midrule
Full CCS & 76\% & 43 \\
\rowcolor{green!15}
\alert{Partial CCS (BFG)} & \alert{36\%} & \alert{28} \\
\bottomrule
\end{tabular}
\end{table}
\pause
\vspace{0.5cm}
\alert{28 €/t = \textbf{cheaper than a tank of petrol}}\\
\alert{Deployable \textbf{now} with existing tech}
\end{frame}
%====================================

%====================================
\begin{frame}{Future: Hydrogen Steel (H$_2$-DRI + EAF)}
\centering
\includegraphics[width=0.7\linewidth]{hydrogen_steel_simple.pdf}\\
\tiny Hybrit, 2024
\vspace{0.3cm}
\small
\begin{itemize}
    \item Replace coal with \textbf{green hydrogen}
    \item CO$_2$ $\rightarrow$ \textbf{H$_2$O} (water!)
    \item \textbf{Near-zero emissions}
    \item \alert{3 plants under construction}: Sweden, Germany
\end{itemize}
\end{frame}
%====================================

%====================================
\begin{frame}{Roadmap to Zero-Emissions Steel}
\small
\begin{enumerate}
    \item \textbf{Now (2025–2030)}: \\
       \alert{Partial CCS + waste heat} $\rightarrow$ 36\% cut @ 28 €/t
    \item \textbf{2030–2040}: \\
       Scale CCS + scrap-EAF (48\% by 2050)
    \item \textbf{2040+}: \\
       \alert{H$_2$-DRI + green EAF} $\rightarrow$ near-zero
\end{enumerate}
\pause
\vspace{0.5cm}
\centering
\alert{\textbf{No single fix} — we need \textbf{all paths}}
\end{frame}
%====================================

%====================================
\begin{frame}{2025 Tipping Points}
\small
\begin{itemize}
    \item \textbf{China}: No new coal-BF permits (H1 2024)
    \item \textbf{20/50 top producers}: Net-zero by 2050
    \item \textbf{H$_2$-DRI plants}: Stegra, thyssenkrupp
    \item \textbf{ING Bank}: No BF/met coal financing
    \item \textbf{India}: Green Steel Taxonomy (Dec 2024)
\end{itemize}
\end{frame}
%====================================

%====================================
\begin{frame}{What You Can Do}
\centering
\small
\begin{itemize}
    \item \textbf{Ask}: “What’s your steel’s CO$_2$ footprint?”
    \item \textbf{Support}: Carbon price >60 €/t, green H$_2$
    \item \textbf{Choose}: Recycled steel when possible
\end{itemize}
\pause
\vspace{0.5cm}
\Large \alert{Sustainable steel is possible — \\
and it starts \textbf{today}}
\end{frame}
%====================================

%====================================
\begin{frame}{References}
\tiny
\begin{itemize}
    \item SteelWatch, \textit{Why steelmaking drives climate change} (Jan 2025)
    \item World Steel Association, \textit{World Steel in Figures 2025}
    \item Ramboll/Climate Group, \textit{Steel \& Concrete Survey 2024}
    \item Biermann et al., \textit{IJGGC} \textbf{91}, 102833 (2019)
    \item Porter et al., \textit{TCCS-11} (2021) – CO₂ purity
    \item IEA Net Zero by 2050; Hybrit 2024
\end{itemize}
\end{frame}
%====================================

\begin{frame}[allowframebreaks]{References}
\tiny
\printbibliography
\end{frame}

%====================================
\begin{frame}{Questions?}
\centering
\Huge Any questions?\\
\vspace{1cm}
\Large Thank you!
\end{frame}
%====================================

\end{document}
